\unnumberedchapter{Abstract} 
\chapter*{Abstract} 
% \subsection*{\thesistitle}

The speed of process and the amount of data to be used in defending the cyber space cannot be handled by humans without considerable automation at the cloud base systems.
However, it is difficult to develop software with conventional fixed algorithms (hard-wired logic on decision making level)
for effectively defending against the dynamically evolving attacks in networks. This situation can be handled by applying methods of artificial intelligence that provide flexibility
and learning capability to software.

Computer Security system providers are unable to provide timely security update. Most security systems are not designed to be adaptive to 
the increasing number of new threats. Companies lose considerable amount of time and resources when security attacks manifest themselves. 

As a answer to these problems this research is aimed at developing security systems capable of learning and updating themselves.

The goal is to create security systems that will autonomously mature with expose to threats over time.
To achieve this goal this research is proposing artificial intelligence based security systems with learning capability 
to perform intrusion, detection, repair and agnostics and defending network.

% Global business internet trafic is expected to increase three-fold from 2017 to 2022. Cyber analysts are finding it increasingly difficult to 
% effectvely monitor current levels of data volume, velocity and variety acras firewalls signature based cybersecurity solutions.
