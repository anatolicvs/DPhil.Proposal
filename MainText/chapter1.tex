
\chapter{Introduction} \label{ch-1}

\section{AI: The next Frontier in IT Security} 

\textit{AI} in cybersecurity is a set of capabilities that allows organizations to detect, predict and respond to cyber threats in real-time
using machine and deep learning.


\textbf{AI-enabled cybersecurity is increasingly necessary}, organizations face an urgent need to continually ramp up and improve their cybersecurity. 

This is because the number of end user devices, networks and user interfaces continues to grow as a result of advances in cloud, the IoT,5G and conversational interfaces.

The increases the difficulty involved with administering a computer network. Having artificially intelligent network infrastructure that can learn to detect,
report and repair network security problems is an advantage to network administrators. 

Adaptability is another key reason that brought AI and computer security close to each other. The malicious entities that generate computer security attacks have gained intelligence. 
The type of attacks evolved from a simple password guessing to a staged and distributed network attack that can be equipped with a stealth mode and attack that mutation.
The compute security field has to adopt to rapid change and the evolution of security attacks. AI us well situated to answer this situation. AI is concerned with creating systems that evolve
and react depending on the changes in the surrounding environment. 

\textit{Fighting the Unknown}. For many years, the efforts of the IT security industry were based on the assumptions 
that attacks were conducted on a large scale and major threats spread before arriving at a specific network or computer.
Solutions were created on the basis of these assumptions and addressed as fellows. A typical threat was identified as a virus, which was investigated 
by IT security company labs that identified the virus signature and sent this information to endpoints installed with leading antivirus software. 

\textit{Internet of Things} In the past, most Internet interactions were interactions between humans that communicated via different platforms over the global network. 
Communications between humans, as the primary users of online communications will change dramatically with the evolution of the IoT realm ~\cite{Ko:2016:STU:2909066.2835492}. 

Most of the entities that will communicate on the Web in the future will be machines, and they will be used to initiate 
reports, make contact, or respond to requests. ~\cite{Komninos:2012}
The means to facilitate this is characterized by easy access to the Web and the ability for mass distribution. For example, there is a possibility to turn households into smart homes with 
intercommunicative technology instead of investing in designing a supporting infrastructure. 
Even scattering sensors along oil fields is the result of technological developments that led to a decrease in 
the price of materials, easy logistical operation, and resistant products that meet industrial standards. ~\cite{Chen:2017:DAI:3046067.3046227}

The cyber world has barely begun to focus on new types of dangers such as the autonomous world, since this process has just started. However, the severity 
level of the threat is clear. In coming years, the cyber defense community will be called upon to develop 
new and effective technologies to meet the challenges posed by our changing and increasingly autonomous world.



\section{Research Plan}

 This should begin with the specific aims of the research and provide a concrete plan for completion of the research including the design and methods. This section should include an explanation of how the methods will address the aims and the significance of the results for the field.

\section{Progress Report}

This should be a report on the research achievements of the student in the laboratory of the proposed supervisor during Preliminary Thesis Research. 
The report should not duplicate material previously submitted for evaluation as part of a previous degree, but may include work completed during rotations at OIST. 
The report may include examples of results obtained with the methods proposed. It is understood that results may not be available in projects 
requiring, for example, development of methods, sample preparation, or recruitment of participants, in which case other evidence of progress should be reported.


