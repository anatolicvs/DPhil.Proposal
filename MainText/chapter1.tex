
\chapter{Introduction} \label{ch-1}

In recent years, a large amount of research addressing the contribution of intelligent
systems to cyber security has been conducted. One can hear more and more about artificial intelligence (AI), machine learning, and deep learning systems in cyber security.
As often happens with buzz words, nonexperts tend to use them casually and loosely.
Despite the fact that the borderlines between such terms are not always clear and these
borderlines become even fuzzier with new practical and theoretical developments, it is
important to try to characterize the focus of these systems in the field.


\section{AI: The next Frontier in IT Security} 

\textit{AI} in cybersecurity is a set of capabilities that allows organizations to detect, predict and respond to cyber threats in real-time
using machine and deep learning.


\textbf{AI-enabled cybersecurity is increasingly necessary}, organizations face an urgent need to continually ramp up and improve their cybersecurity. 

This is because the number of end user devices, networks and user interfaces continues to grow as a result of advances in cloud, the IoT,5G and conversational interfaces.

The increases the difficulty involved with administering a computer network. Having artificially intelligent network infrastructure that can learn to detect,
report and repair network security problems is an advantage to network administrators. 

Adaptability is another key reason that brought AI and computer security close to each other. The malicious entities that generate computer security attacks have gained intelligence. 
The type of attacks evolved from a simple password guessing to a staged and distributed network attack that can be equipped with a stealth mode and attack that mutation.
The compute security field has to adopt to rapid change and the evolution of security attacks. AI us well situated to answer this situation. AI is concerned with creating systems that evolve
and react depending on the changes in the surrounding environment. 

\textit{Fighting the Unknown}. For many years, the efforts of the IT security industry were based on the assumptions 
that attacks were conducted on a large scale and major threats spread before arriving at a specific network or computer.
Solutions were created on the basis of these assumptions and addressed as fellows. A typical threat was identified as a virus, which was investigated 
by IT security company labs that identified the virus signature and sent this information to endpoints installed with leading antivirus software. 

\textit{Internet of Things} In the past, most Internet interactions were interactions between humans that communicated via different platforms over the global network. 
Communications between humans, as the primary users of online communications will change dramatically with the evolution of the IoT realm ~\cite{Ko:2016:STU:2909066.2835492}. 

Most of the entities that will communicate on the Web in the future will be machines, and they will be used to initiate 
reports, make contact, or respond to requests. ~\cite{Komninos:2012}
The means to facilitate this is characterized by easy access to the Web and the ability for mass distribution. For example, there is a possibility to turn households into smart homes with 
intercommunicative technology instead of investing in designing a supporting infrastructure. 
Even scattering sensors along oil fields is the result of technological developments that led to a decrease in 
the price of materials, easy logistical operation, and resistant products that meet industrial standards. ~\cite{Chen:2017:DAI:3046067.3046227}

The cyber world has barely begun to focus on new types of dangers such as the autonomous world, since this process has just started. However, the severity 
level of the threat is clear. In coming years, the cyber defense community will be called upon to develop 
new and effective technologies to meet the challenges posed by our changing and increasingly autonomous world.


\section{The Contribution of Intelligent Systems to Cyber Security}\label{sec:cont-intelligent-systems-cyber-security}

Artificial intelligence (AI), which was officially established in the late 1950s by scholars such as Marvin Minsky, is used as a generic name representing a wide variety of
methods, tools, and techniques that mimic “cognitive” functions or tasks that people
associate with the human mind, such as “learning,” “planning,” “reasoning,” or “problem solving” ~\cite{Russell:1995}.
The importance of AI in cyber security is twofold and related to two opposing directions. The first direction focuses on 
AI-controlled systems as potential targets of cyber attacks, mainly due to their increasing role in controlling vital and complex 
systems. There are numerous examples of cyber risks related to AI-controlled systems, such as smart vehicles ~\cite{Berger:2013:PLC:2489103.2514809}
, smart grids and smart cities ~\cite{ALDAIRI20171086}.

\textbf{Knowledge-based systems and machine learning methods} are two well-known classes of AI methods that contain valuable tools
used in cyber security. In Knowledge-based systems, a huge amount of (experts') Knowledge is uploaded to the computer memory (thus, it is also called expert systems). ~\cite{Felgenbaum:1977:AAI:1622943.1623042}
The learning part in these systems is based on the reasoning related to this large body of knowledge, which is often obtained  by programmed rules (such as "if-then" and "inference logic rules").

Antivirus and antispam software packages ~\cite{Blanzieri:2008:SLT:1612711.1612715} represent straightforward
implementations of expert systems in cyber security. In this case, expert knowledge
(accumulated based on a large amount of transactions) regarding the procedures used—
such as applied protocols, network traffic (e.g., HTTP, HTTPS, VoIP, or email), and I/O
interactions with the operating system—is organized systematically to protect the
users from cyber breaches. There are several papers in this issue that serve as good
examples of such expert systems ~\cite{Maltinsky:2017:NNM:3055535.3040966} .


\textit{Machine-learning methods,} unlike knowledge-based systems, usually refer to applications in which the learning component is performed by the computers “themselves.”
This is often done by extracting relevant patterns from the data and using them to
derive predictions and smart recommendations. In other words, machine-learning algorithms are improved “automatically” through experience with the data, while giving
the computers “the ability to learn without being explicitly programmed,” as suggested
long before the field of cyber security was formally established ~\cite{Samuel:1959:SML:1661923.1661924}. There
are many examples of cyber-security tasks that can be addressed by machine learning,
including user monitoring, spam filtering, zero-day attack identification, risk analysis,
and many more.

It is well known that machine-learning methods are divided into two main classes
(and a hybrid class of methods that combines the two). The first class contains unsuper-
vised learning methods, in which untagged data samples are introduced to the system
in order to find significant patterns. Fraud detection in financial systems, anomaly
detection in communication protocols, and segmentation of both users and software
packages according to their risk potential are good examples of unsupervised machine-
learning methods that apply techniques, such as anomaly detection and clustering, to
identify both “positive” or “negative” deviations from the norm. These deviations are
then mapped into actions that include, for example, risk assessment of new software
packages, blacklisted websites, or blocking Internet connections with high rates of sus-
picious users. Two papers published in this issue are introduced in this section, which
serve as good examples of unsupervised machine-learning methods ~\cite{Maltinsky:2017:NNM:3055535.3040966}
~\cite{Harel:2017:CSR:3055535.3057729}.

The second class of methods belongs to supervised learning, in which the data samples
that are introduced to the system are tagged a priori. In other words, the sample data
inputs are coupled with their desired outputs (thus the term supervised). The goal is to
learn a general rule that maps inputs to outputs. Some examples from the cyber security
domain are users’ risk scores [Ben Neria et al. 2017], for which descriptive features
of users—such as the communication volume, time, and the type of interaction—are
tagged either as “risky” or “nonrisky” and are then learned by the system to predict
high-risk users in advance [Gruber and Ben-Gal 2017].

\section{Research Plan}

 As an aims of the research and We provided a concrete plan for completion of the research including the design and methods. 

\begin{itemize}
    \item {
        Analyzing previous AI solutions which is developed for cybersecurity and figure out the deficiencies of the application. 
     }
    
     \item{
        Investigating "exploit databases" and collecting data related the kind of the threats.  
     }
     \item {
        Analyzing Cloud Based Systems' network infrastructures for collecting data. 
     }
    
     \item {
        Researching common and recent threats for the computer networks.
     }

     \item {
        Investigating the literature of the machine learning algorithms. 
    }
    \item {
        Analyzing the vulnerabilities of the cloud systems.   
    }

     \item {
        Defining the constraints of the problem.
     }

     \item {
        Modeling the problem. 
     }
     \item {
        Creating the fittest machine learning models for AI application 
     }
     \item {
        Developing the AI application that can detect, repair and defense the cloud based systems to against the cyber threats and attacks.
     }
    \item {
        Creating real-world test environment and creating real-time cyber-attacks scenarios for test the developed AI supported security applications.
    }

    \item {
        Regarding the test results of the application, optimize the AI algorithms with heuristic and metaheuristic approaches.  
    }

\end{itemize} 